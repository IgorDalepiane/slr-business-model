\begin{table*}[!htb]
\renewcommand{\arraystretch}{1.2}
\caption{Modelos de Negócio de Software Identificados na Revisão Sistemática}
\label{tab:extracted-business-models}
\centering
\small
\begin{tabular}{p{5cm}p{9cm}r}
\toprule
\textbf{Modelo de Negócio} & \textbf{Descrição} & \textbf{Menções} \\
\midrule
\rowcolor{gray!15} Platform business model & Modelo que cria valor através de ecossistemas de conexões externas e dados & 391 \\
SaaS (Software as a Service) & Software entregue como serviço hospedado e acessado via Internet & 289 \\
\rowcolor{gray!15} Freemium model & Modelo que oferece versão básica gratuita com recursos premium pagos & 264 \\
Open source business model & Modelo baseado em software de código aberto com diversas estratégias de comercialização & 188 \\
\rowcolor{gray!15} Subscription-based pricing & Modelo de pagamento recorrente com cobrança mensal/anual & 153 \\
Cloud delivery model & Distribuição de software através de infraestrutura em nuvem & 152 \\
\rowcolor{gray!15} Service-oriented model & Modelo que enfatiza entrega de serviço sobre venda de produto & 144 \\
Data-driven monetization & Geração de receita a partir de dados e analytics de usuários & 108 \\
\rowcolor{gray!15} Hybrid business model & Modelo que combina múltiplas abordagens (ex: SaaS + licenciamento) & 77 \\
Marketplace model & Modelo que facilita transações entre múltiplas partes & 63 \\
\rowcolor{gray!15} API-based business model & Modelo que entrega valor através de interfaces de programação & 44 \\
Pay-as-you-go pricing & Precificação baseada em consumo/uso & 40 \\
\rowcolor{gray!15} Value-based pricing & Precificação baseada no valor percebido pelo cliente & 33 \\
Perpetual licensing & Licenciamento tradicional com pagamento único & 14 \\
\bottomrule
\end{tabular}
\vspace{0.2cm}

\footnotesize\textit{Nota: Menções referem-se ao número de quotes classificadas com cada código. Um mesmo estudo pode mencionar múltiplos modelos.}
\end{table*}
